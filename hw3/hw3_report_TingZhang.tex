%----------------------------------------------------------------------------------------
%	PACKAGES AND OTHER DOCUMENT CONFIGURATIONS
%----------------------------------------------------------------------------------------

\documentclass[11pt]{article}
%\input{mydef.tex}
\usepackage{fancyhdr} % Required for custom headers
\usepackage{lastpage} % Required to determine the last page for the footer
\usepackage{extramarks} % Required for headers and footers
\usepackage[usenames,dvipsnames]{color} % Required for custom colors
\usepackage{graphicx} % Required to insert images
\usepackage{listings} % Required for insertion of code
\usepackage{courier} % Required for the courier font
\usepackage{amssymb,amsmath}
\usepackage{amsfonts}
\usepackage{mathtools}
\usepackage{subfigure}
\usepackage{enumitem}
\usepackage{bm}
\usepackage{url}
\usepackage[stable]{footmisc}
\usepackage{booktabs}
\usepackage[square]{natbib}
\usepackage{indentfirst}
%\usepackage[colorlinks, linkcolor=red, anchorcolor=purple, citecolor=blue]{hyperref}
\usepackage{hyperref}

\usepackage{multicol}
\setlength{\columnsep}{1cm}

% Margins
\topmargin=-0.45in
\evensidemargin=0in
\oddsidemargin=0in
\textwidth=6.5in
\textheight=9.0in
\headsep=0.25in
\setlength{\headheight}{13.6pt}
\linespread{1.1} % Line spacing

% Set up the header and footer
\pagestyle{fancy}
\lhead{CS 57800} % Top left header
\chead{}
\rhead{Homework 3} % Top right header
\lfoot{} % Bottom left footer
\cfoot{} % Bottom center footer
\rfoot{Page\ \thepage\ of\ \protect\pageref{LastPage}} % Bottom right footer
\renewcommand\headrulewidth{0.4pt} % Size of the header rule
\renewcommand\footrulewidth{0.4pt} % Size of the footer rule

\renewcommand*{\thefootnote}{\fnsymbol{footnote}}

\setlength{\parskip}{.2\baselineskip}
%\setlength\parindent{0pt} % Removes all indentation from paragraphs

\title{
\textbf{CS57800 Statistical Machine Learning} \\ \textsc{Homework 3} \\
\normalsize\vspace{0.1in}
}

\author{
	\textbf{Ting Zhang} \\
	School of Industiral Engineering\\
	\texttt{zhan1013@purdue.edu}
}

\date{\today}
%----------------------------------------------------------------------------------------

\begin{document}

\maketitle
%\thispagestyle{empty}

\section{Foundations}
\begin{itemize}
\item[1.]
The VC dimension of C is \(2n+1\).
\item[2.] To show it is a convex function, we can compute the second derivative of the function.
\begin{align*}
& \frac{dl}{d\hat{y}} = \frac{1}{\log 2}[\frac{-ye^{-y\hat{y}}}{(1+e^{-y\hat{y}}) \log 2}] = \frac{1}{(\log 2)^2} (-1 + \frac{y}{1+e^{-y\hat{y}}})  \\
& \frac{d^2l}{d\hat{y}^2} =  \frac{1}{(\log 2)^2}[0 + y*(-2)*(\frac{1}{1+e^{-y\hat{y}}})^2 * (-y) * e^{-y\hat{y}}] = \frac{2y^2e^{-y\hat{y}}}{(\log 2)^2 (1+e^{-y\hat{y})}} > 0
\end{align*}
Since the second derivative of this function is greater than 0, then this function is a convex function.
\item[3.] The training error of the final hypothesis can be formulated as
\begin{equation*}
TrainingError(H_{final}) \leq \prod_{t} [2 \sqrt{\epsilon_t(1-\epsilon_t)}].
\end{equation*}
\item[4.] The decision boundary of each weak hypothesis is a axis-parallel line, as shown in Figure~\ref{•}. Positive points are represented with "+", and negative points are represented with "*". Then the process of doing Adaboost is as following: \\
(1) First iteration \(t=1\), find the decision boundary that makes the least number of mistakes. In first iteration, each example is uniformlly distributed, with \(D_1(i)=0.1\). Here it found \(x_1>6\), with 2 mistakes, including example 9 and 10. Therefore, the error is \(\epsilon_1=0.2\). The weak hypothesis is \(h_1(x_i): sign(x_1-6)\).\\
(2) Second iteration \(t=2\), first, update the distribution of each example. To update the distribution, we need to follow these euqations:
\begin{equation*}
\alpha_1 = 0.5 * \ln (\frac{1-\epsilon_1}{\epsilon_1}) = 0.5 * \ln (\frac{1-0.2}{0.2}) = 0.693
e_{-\alpha_1} = 0.500 
e_{\alpha_1} = 2.000
\end{equation*}
So, the distribution of the correctly classified examples \(i=1,2,3,4,5,6,7,8\) becomes
\begin{equation*}
D_2(i) = \frac{D_1(i) *e_{-\alpha_1} }{Z_1} = \frac{0.05}{Z_1}
\end{equation*}
and the distribution of the wrongly classified examples \(i=9,10\) becomes
\begin{equation*}
D_2(i) = \frac{D_1(i) *e_{\alpha_1} }{Z_1} = \frac{0.2}{Z_1}
\end{equation*}
Since \(\sum_{i=1}^{10} D_2(i) = 1\), we have \((8*0.05+2*0.2)/Z_1=1\). So, \(Z_1 = 0.8\). Therefore, the distribution of the examples now become
\begin{equation*}
D_2(i) =
  \begin{cases}
    0.0625       & \quad \text{if } i=1,2,3,4,5,6,7,8\\
    0.25           & \quad \text{if } i=9,10\\
  \end{cases}
\end{equation*}
Then, finding the weak hypothesis that has the smallest error \(\sum_{i=1} D_2(i)\), where \(i\) is the index of wrongly classified example. Then, it find the weak hypothesis to be \(h_2(x_i): sign(x_2-8)\), with four mistakes, including example 3, 6, 7, 9. And the error is \(\epsilon_2 = 0.25\). We can also compute \(\alpha_2 = 0.5 * \ln (\frac{1-\epsilon_2}{\epsilon_2}) = 0.549 \). \\
Then, the final hypothesis is 
\begin{equation*}
H_{final}(x) = sign(\sum_{t} \alpha_t h_t(x)) = sign(0.693*sign(x_1-6) + 0.549*sign(x_2-8))
\end{equation*}
\item[5.] To verify a kernel function, we need to proof the positive semi-definite property of it.
\begin{align*}
& \iint f(\vec{x})K(\vec{x},\vec{y})f(\vec{y})d\vec{x}d\vec{y} \\
& =\iint f(\vec{x})[\alpha K_1(\vec{x},\vec{y})+\beta K_2(\vec{x},\vec{y})]f(\vec{y})d\vec{x}d\vec{y} \\
& = \iint \alpha f(\vec{x})K_1(\vec{x},\vec{y})f(\vec{y})d\vec{x}d\vec{y} + \beta f(\vec{x})K_2(\vec{x},\vec{y})f(\vec{y})d\vec{x}d\vec{y} \\
& > 0*0 + 0*0 \\
& = 0
\end{align*}
Because, \(K_1(\vec{x},\vec{y})\) and \(K_2(\vec{x},\vec{y})\) are positive semi-definite, \(\alpha\) and \(\beta\) are all positive.
\item[6.] \(\xi\) is defined as the margin of the soft SVM. When an example is classified correctly, \(0 \leq \xi \leq 1\); while if an example is classified wrongly, then \(\xi > 1\). Therefore, with \(M\) examples, there are at most \(M\) mistakes. Since  \(\xi > 1\), then \(\sum_{i=1}^{M} \xi_i > M*1 = M\). Therefore, \(\sum_{i=1}^{M} \xi_i\) is the upper bound on the training error of the classifier.
\end{itemize}

%x2>8: 0.25
%10:0.38
%8:0.476

\section{Programming Report}
...

%\nocite{*}
%\bibliographystyle{plainnat}
%\bibliography{all}

\end{document}
